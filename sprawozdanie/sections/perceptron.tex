\documentclass[../main.tex]{subfiles}

\begin{document}
    \section{Badania wpływu wartości wag początkowych na szybkość uczenia}
    \paragraph{}
    Badanie ma na celu zbadanie wpływu wartości wag początkowych na działanie perceptronu prostego oraz Adaline. Badania przeprowadzone zostały dla unipolarnej funkcji aktywacji oraz zbioru testowego. Wykorzystano współczynnik uczenia a = 0.05.
    
    \paragraph{}
    Zbadano natomiast następujące przedziały:
    \begin{itemize}
     \item (-1, 1)
     \item (-0.8,  0.8)
     \item (-0.5, 0.5)
     \item (-0.2, 0.2)
     \item (0, 0)
    \end{itemize}
    
    \paragraph{}
    Tak jak zostało to wcześniej wspomniane, prezentowane wyniki są wartościami uśrednionymi, \\hline
    uzyskanymi w skutek wielokrotnego uruchomienia algorytmu i prezentują się one następująco.
    \newline 
    \begin{figure}[H]
    \centering
    \begin{tikzpicture}

    \pgfplotsset{
        scale only axis,
    }

    \begin{axis}[
        xlabel=Wartość bezwględna wag początkowych,
        ylabel=Liczba epok,
    ]
        \addplot[only marks, mark=x]
        coordinates{
        (1,           64)
        (0.8,    43)
        (0.5,       39)
        (0.2,       30)
        (0.05,       27)
        (0,            26)
        }; \label{scatter_weights_simple_10}

        % plot 1 legend entry
        \addlegendimage{/pgfplots/refstyle=plot_one}
    \end{axis}
    \end{tikzpicture}
    \caption{Wyniki badań uzyskane w skutek 10 uruchomień}
    \end{figure}

    \begin{figure}[H]
    \centering
    \begin{tikzpicture}

    \pgfplotsset{
        scale only axis,
    }

    \begin{axis}[
        xlabel=Wartość bezwględna wag początkowych,
        ylabel=Liczba epok,
    ]
        \addplot[only marks, mark=x]
        coordinates{
        (1,           61)
        (0.8,    43)
        (0.5,       30)
        (0.2,       25)
        (0.05,       26)
        (0,            26)
        }; \label{scatter_weights_simple_50}

        % plot 1 legend entry
        \addlegendimage{/pgfplots/refstyle=plot_one}
    \end{axis}
    \end{tikzpicture}
    \caption{Wyniki badań uzyskane w skutek 50 uruchomień}
    \end{figure}
    
    \paragraph{}
    W przeprowadzonych badaniach można zauważyć wzrost liczby epok wymaganych do dobrania odpowiednich wag, wraz ze wzrostem wielkości przedziału losowanych wag. Wynika to z faktu, że w przypadku dużych przedziałów wag mogą one wylosować skrajnie różne wartości. Duża różnica między początkową, a optymalną wartością wagi prowadzi, przy stałym współczynniku uczenia, do wzrostu wymaganej liczby epok, a co za tym idzie czasu wymaganego na ukończenie treningu.
    
    \paragraph{}
    Dalsze badania ujawniły 
    
    
\end{document}
