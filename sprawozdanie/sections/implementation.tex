\documentclass[../main.tex]{subfiles}

\begin{document}
    \paragraph{}
    Wykorzystywane w badań oprogramowanie zostało zaimplementowane w języki \textbf{Python 3.6} z wykorzystaniem biblioteki \textbf{NumPy}, wykorzystaniej do przeprowadzenia operacji na macierzach i wektorach. Dodatkowo wykorzystany został moduł \textbf{csv} w celu zautomatyzowania przeprowadzanych testów. W procesie tworzenia wykorzystane zostało środowisko \textbf{Visual Studio Code}.
    
    \paragraph{}
    Implementacje obu rodzajów perceptronów zostały zawarte w dwóch plikach \emph{simpleperceptron.py} oraz \emph{adaline.py}. Cała mechanika testowania została zaimplementowana natomiast w pliku \emph{testing.py}. W wyniku testów wygenerowany został plik \emph{results.csv} zawierający wyniki przeprowadzonych badań.
    
    \paragraph{}
    Konstruktory klas reprezentujących poszczególne rodzaje perceptronów przedstawiają się następująco.
    
        \begin{verbatim}
    SimplePerceptron(numOfInputs, (minWeight, maxWeight),
                     activation, activationThreshold)
    \end{verbatim}

    \begin{verbatim}
    Adaline(numofinputs, (minweight, maxweight), activationthreshold)
    \end{verbatim}
    
        Wyjaśnienia poszczególnych parametrów przedstawiają się następująco:
    \begin{itemize}
     \item numOfInputs - liczba danych treningowych.
     \item minWeight - minimalna wartość wag początkowych.
     \item maxWeight - maksymalna wartość wag początkowych.
     \item activation - funkcja aktywująca (bipolarna lub unipolarna).
     \item activationThreshold - wartość graniczna funkcji aktywującej.
    \end{itemize}

    Funkcje rozpoczynające proces treningu natomiast prezentują się następująco:
    
    \begin{verbatim}
    SimplePerceptron.train(trainingData, labels, iterations, learning_rate)
    \end{verbatim}
    
    \begin{verbatim}
    Adaline.train(trainingData, labels, iterations, errorThreshold, learningRate)
    \end{verbatim}
    
    Wyjaśnienia poszczególnych parametrów przedstawiają się następująco:
    \begin{itemize}
     \item trainingData - wektor danych wejściowych.
     \item labels - wektor danych oczekiwanych.
     \item iterations - maksymalna, nieprzekraczalna liczba epok.
     \item errorThreshold - minimalna wartość błędu średniokwadratowego (dla perceptronu Adaline).
     \item learningRate - współczynnik uczenia.
    \end{itemize}

\end{document}
