\documentclass[../main.tex]{subfiles}

\begin{document}
    \paragraph{}
    Celem przeprowadzanych badań było omówienie właściwości i zachowania metod uczenia realizowanych przez pojedyńczy neruon oraz zbadanie wpływu hiperparametrów na szybkość uczenia neuronu zarówno dla perceptronu prostego jak i Adaline.
    
    \paragraph{}
    W trakcie badań zbadano wpływ różnych hiperparametrów na proces uczenia perceptronu prostego i Adaline. Podstawową miarą szybkości procesu trenowania jest liczba epok, które były wymagane do ustalenia odpowiednich wag.
    W przypadku perceptronu prostego jest to liczba epok, po których wagi na połączeniach między wejściem, a wyjściem neuronu przestają być aktualizowane. W przypadku perceptronu Adaline warunkiem kończącym trening jest osiągnięcie błędu średniokwadratowego mniejszego od zadanej wcześniej wartości. W celu przyśpieszenia badań wprowadzono także ograniczenie polegające na przerwaniu treningu po upływie 1000 epok. Wszystkie otrzymane wyniki są wartościami uśrednionymi uzyskanymi w skutek wielokrotnego uruchomienia algorytmu.
    
    \paragraph{}
    Badania perceptronu prostego zostały przeprowadzone zarówno dla unipolarnej jak i bipolarnej funkcji przejścia. W przypadku perceptronu Adaline, ze względu na jego charakterystykę, wykorzystano tylko bipolarną funkcję przejścia. Uczenie neuronu zostało przeprowadzone na zbiorze treningowym. Walidacja poprawności predykcji została natomiast przeprowadzona z użyciem zbioru walidacyjnego. Postać wykorzystywanych danych prezentuje się następująco:
    
    \begin{table}[h]
    \centering
    \begin{tabular}{|| c | c | c ||}
    \hline
    x1 & x2 & y \\ \hline
    1  & 1  & 1 \\
    1  & 0  & 0 \\
    0  & 1  & 0 \\
    0  & 0  & 0 \\ \hline
    \end{tabular}
    \caption{Zbiór danych dla unipolarnej funkcji aktywacji}
    \label{dataset_unipolar:1}
    \end{table}
    
        \begin{table}[h]
    \centering
    \begin{tabular}{|| c | c | c ||}
    \hline
    x1 & x2 & y \\ \hline
    1  & 1  & 1 \\
    1  & 0  & -1 \\
    0  & 1  & -1 \\
    0  & 0  & -1 \\ \hline
    \end{tabular}
    \caption{Zbiór danych dla bipolarnej funkcji aktywacji}
    \label{dataset_unipolar:1}
    \end{table}
    
    W powyższych tabelach para \textbf{(x1, x2)} oznacza dane wejściowe, natomiast kolumna \textbf{y} zawiera wartości oczekiwane.
\end{document}
