\documentclass[a4paper,10pt]{article}

\usepackage{titlesec}
\usepackage{polski}
\usepackage[utf8]{inputenc}
\usepackage[document]{ragged2e}
\usepackage{geometry}
\usepackage{listings}
\usepackage{makecell}
\usepackage{float}
\usepackage{siunitx}
\usepackage{pgfplotstable}
\usepackage{subfiles}
\usepackage{graphicx}
\usepackage{tikz}
\usepackage{pgfplots}
\pgfplotsset{width=7.5cm,compat=1.12}
\usepgfplotslibrary{fillbetween}

\pgfplotsset{compat=newest}
\usepgfplotslibrary{units}

\pdfinfo{%
  /Title    (Sieci neuronowe Ćwiczenie 1)
  /Author   (Przemysław Pietrzak)
  /Creator  (Przemyslaw Pietrzak)
  /Producer (Przemysław Pietrzak)
  /Subject  (Sieci neruonowe)
  /Keywords (sztuczna, inteligencja)
}

\lstset{frame=tb,
  language=Java,
  aboveskip=3mm,
  belowskip=3mm,
  showstringspaces=false,
  columns=flexible,
  basicstyle={\small\ttfamily},
  numbers=none,
  breaklines=true,
  breakatwhitespace=true,
  tabsize=3
}

\sisetup{
  round-mode          = places,
  round-precision     = 2,
}

\begin{document}
    \begin{titlepage}
     \newgeometry{centering, margin=1.5cm}
     \vspace*{\fill}
    
     \vspace*{-4cm}
     \centering
     \Huge\bfseries\
     {Sieci neuronowe}
    
     \LARGE
     \centering
     \vspace{2cm}
     {Sprawozdanie nr 1}
    
     \Large
     \centering
     {Perceptron prosty i Adaline}
     
     \vspace*{0.5cm}
     
     \centering
     \large 
     \vspace{0.5cm}
     Przemysław Pietrzak, 238083
     
     Środa, 14:15
     
     \vspace*{\fill}
     \restoregeometry
    \end{titlepage}
    
    \newpage
    \tableofcontents
    
    \newpage
    \justify
    \section{Plan eksperymentów}
    \subfile{sections/plan}
    
    \newpage
    \justify
    \section{Wpływ hiperparameterów na szybkość uczenia perceptronu prostego}
    \subfile{sections/perceptron}
    
    \newpage
    \justify
    \section{Wpływ hiperparameterów na szybkość uczenia perceptronu Adaline}
    \subfile{sections/adaline}
    
    \newpage
    \justify
    \section{Wnioski końcowe}
    \subfile{sections/summary}
    
\end{document}
